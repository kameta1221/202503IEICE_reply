\documentclass[a4j]{jsarticle}
\usepackage{txfonts}
\usepackage{ascmac}
\renewcommand{\labelenumi}{\textbf{\theenumi . }}
\title{\bf \Large Reply Letter}
\author{Masahiro Nishimura, Taito Manabe, and Yuichiro Shibata}
\date{}

\begin{document}

\maketitle

We deeply appreciate your careful review and valuable suggestions.  
We have carefully revised the paper according to the reviewers' 
comments as follows.
In the revised manuscript, corrections were made The revision also includes a typo correction as shown in ``Other Modifications'' section. All the modifications are indicated by red color in the revised manuscript. Changes in figures are indicated by making their captions red.
%  to address the comments and additional typo corrections were made.
% In the revised manuscript, corrections corresponding to the comments are indicated in red.


\section*{Reply to Associate Editor}
\begin{screen}
Thank you for your submission to IEICE Transactions. Both of our reviewers
think that the paper is interesting and could be published after the revision. 

"Mandatory:" parts by two reviewers are the mandatory conditions
of the conditional acceptance, and other parts are optional.
Please revise based on their comments.
\end{screen}
Thank you very much for the opportunity to revise 
our manuscript submitted to IEICE Transactions.
We sincerely appreciate the reviewers’ constructive and encouraging comments.
We are pleased to know that both reviewers found our paper interesting and 
potentially publishable after revision.
We have carefully addressed not only all the mandatory points raised by both reviewers but also the optional comments to further improve the quality of the manuscript.
% We will carefully address all the ``Mandatory'' points raised by both reviewers, 
% as well as consider the optional suggestions to further improve the quality of 
% the manuscript.


\section*{Reply to Reviewer A}

\begin{screen}
This manuscript proposes an estimation method of FPGA's HW resource for real-time HDR synthesis.
The method estimates the required BRAM resource usage with reasonable accuracy (around 10\% error)
and the authors demonstrate that it is useful for the design space exploration
under the tradeoff between HW resources and image quality.

The manuscript is written in good organization and good English.
However, there are several points that need to be updated.
\end{screen}

\vspace{0.3cm}
\begin{screen}
Mandatory:

Section 3.3 in p.3 says "the compression is used only after the fifth layer".
I think it means that, for instance, the 7-layer configuration implements a compression circuit only for 6th and 7th layers and not for 1st to 5th layers. (i.e. partially compressed and partially not-compressed in the implementation)
However, Section 4.2 in p.5 says "in the modeling, ... compression is performed in all the layers ... to simplify the equation".
Why the model did not estimate layers before and after the fifth layer separately? It seems not so difficult to separately estimate the resources.
\end{screen}
As you correctly pointed out, the actual hardware implementation applies compression 
only to the layers after the fifth one, resulting in a partially compressed architecture. 
However, in Sec. 4.2, we chose to simplify the resource estimation model by assuming that all layers 
are compressed, in order to keep the equation tractable and easy to use for design space exploration.
To clarify this point and show that we have indeed considered the layer-wise breakdown, 
we have added a detailed derivation in the Appendix, where both compressed and uncompressed parts 
are modeled separately. This includes the resource usage for each layer and explains how the simplified 
model was derived based on assumptions about dominant contributors and trade-offs between accuracy and model complexity.
% ご指摘の通り、実際のハードウェア実装では、5層目以降にのみ圧縮が適用されるため、部分的に圧縮されたアーキテクチャとなります。 しかし、セクション4.2では、式を扱いやすくし、設計空間の探索に使いやすくするために、すべてのレイヤーが圧縮されると仮定してリソース推定モデルを単純化することにしました。

% この点を明確にし、実際にレイヤーごとのブレークダウンを考慮したことを示すために、付録で詳細な導出を追加しました。 これには、各レイヤーのリソース使用量が含まれ、支配的な寄与に関する仮定と、精度とモデルの複雑さのトレードオフに基づいて、簡略化されたモデルがどのように導出されたかを説明しています。


\vspace{0.3cm}
\begin{screen}
Mandatory:

In p.5, equation (9) - (12) contains many "magic numbers" such as 7, 3, 67, 4, 11, and so on.
Since this part (the estimation method) would be the key of the proposed method, a detailed explanation is required.
If the rationale for the numbers in the formulae is not provided, it is difficult to determine whether they are correct or not.
\end{screen}
We have added a comprehensive derivation of Eq. (9) - (12) in the Appendix, 
where each constant (e.g., 7, 3, 67, 4, 11, etc.) is explained based on the buffer sizes, 
bit widths, pipeline delays, and module structures described in Sec. 3. 
The derivation is shown step by step using the same variables and architecture diagrams.
% to provide a transparent and reproducible basis for the estimation model.
% We believe this addition enhances the credibility and clarity of the proposed method.

\vspace{0.3cm}
\begin{screen}
Mandatory:

Are W, G, and L in equation (9) the BRAM usage for all layers or single layer? (Since 'd' is the number of processing layers, it implies the equation is for single layer, right?)
\end{screen}
Eq. (9) represents the total BRAM usage across all layers. 
The variable $d$ denotes the number of layers, and the formula incorporates 
the cumulative usage of modules $W$, $G$, and $L$ over the entire pipeline.
% We agree that it may seem as though the equation is for a single layer due to the appearance of $d$, and 
We have revised the second sentence in Sec. 4.1 to clarify this point. 
% While it is possible to formulate BRAM usage per layer, 
% the resulting expression becomes considerably more complex.
% To address this concern, we have added a detailed derivation of Eq. (9) - (12) in the Appendix, 
% showing how the total BRAM usage is computed from per-layer contributions.

\vspace{0.3cm}
\begin{screen}
Mandatory:

* Typos and confusing sentences
\begin{enumerate}
\item What is the definition of 'n' in equation (9) - (12)? Is it the number of input images?

\item Explanation of Fig. 7 is expected. The manuscript says "See [10] for detailed algorithm and implementation" in p.3. However, since you have included the figure, it is worth explaining a bit more.
According to [10], it is ADPCM. Then, at least, the word "ADPCM" should be in the manuscript. (P.3 says "adaptive differential coding". Is it different from ADPCM?)

\item The x-axis of Fig. 8 and Fig. 9 is "Depth". Is it "\# of layers"?

\item Section 5.3 in p.6: "is requires 952.7" -$>$ "requires 952.7"?

\item Section 5.3 in p.6: "The graph also shows that the increase in BRAM ..." -$>$ Which graph? Do you mean the table (Table 1)?

\item Section 5.4 in p.7: "we can easily find that the 6-layer and 7-input configuration achieves the best quality..."
-$>$ I think 5-input is better quality (MEF-SSIMd) than 7-input for the series of 6-layer (orange bars).

\end{enumerate}
\end{screen}

\noindent
We are grateful for your careful review and showing typos and
other problems. For each of the listed items, we have revised
the manuscript as follows:
\begin{enumerate}
  \item $\mathit{n}$ is the number of input images, and the definition of $\mathit{n}$ has been added to the sentence immediately before Eq. (9).
  \item We have added a brief explanation of ADPCM used in Fig. 7, in the first paragraph in Sec. 3.4.
  \item As you pointed out, the x-axis in Fig. 8 and Fig. 9 shows the number of layers. We revised the x-axis labels in Fig. 8 and Fig. 9.
  \item The phrase “is requires 952.7” has been replaced with “requires 952.7.”
  \item As you pointed out, we mean Table 1 and Table 2, not graph. The sentence “The graph ...” has been replaced with “ Table 1 and Table 2 … .” 
  \item Thank you for pointing it out, it was typo. We revised to the correct configuration``7-layer and 6-input''.
\end{enumerate}

\begin{screen}
Optional:

In p.2, right-column: "Mertens' algorithm ... generates a Gaussian pyramid of the input image and a Laplacian pyramid of the weights."
However, the previous paragraph says "The Laplacian pyramid can be regarded as a decomposition of an image ..."
Are they consistent?
The caption of Fig. 3 in [1] also says "Laplacian decomposition of the images and a Gaussian pyramid of the weight maps".
[10] also says "a Laplacian pyramid of the input image and a Gaussian pyramid of the weights".
\end{screen}
The explanation in the original manuscript was incorrect. 
The words ``Laplacian'' and ``Gaussian'' were reversed. 
We revised the sentence.

\vspace{0.3cm}
\begin{screen}
Section 5.2 in p.5: "Resource usage ... is also shown in Fig. 9 as in Fig. 8"
-$>$ I think "as in Fig. 8" is not needed.
\end{screen}
The redundant “as in Fig. 8.” has been removed.

\vspace{0.3cm}
\begin{screen}
Section 5.3 in p.6 says "Since the difference between ... 912 and 1024 is relatively larger than
I am also interested in the resource usage of DSP in FPGA. Are DSPs not critical in the architecture?
\end{screen}
BRAM remains the dominant limiting factor in our system, and DSP usage does not significantly affect the design space exploration.
We have added the explanation to the second sentence of Sec. 4 to clarify this point.

\vspace{0.3cm}
\begin{screen}
Section 5.4 in p.7: Fig. 10 uses "translucent bars", however, it is difficult to distinguish. Please try to improve (ex. adding a dotted-line border to the translucent bars).
\end{screen}
We updated Fig. 10 to improve visibility.

\vspace{0.3cm}
\begin{screen}
In the list of references, the author list of [10] is all capital (e.g. M. NISHIMURA). Please use the same style as other references (ex. M. Nishimura). "adpcm", "hrd" and
"fpga" in the title can be capital letters
\end{screen}
Thank you for your suggestion. The inconsistent capitalization has been corrected.

\section*{Reply to Reviewer B}
\begin{screen}
This paper proposes a method to accurately estimate BRAM usage in HDR compositing on FPGAs using a mathematical model, and is a practical result that contributes greatly to the efficiency of design space exploration. The approach that enables the evaluation of resource consumption for each configuration in advance is solid and useful, and can be evaluated as having engineering significance. On the other hand, the circuit configuration itself is a common configuration in image processing, and its novelty is limited.

The reviewer requests the following corrections. First, there is no quantitative evaluation of the throughput and latency of this implementation. To claim real-time performance, numerical indicators such as processing time and frame rate are essential. Second, the increase in latency is cited as the reason for avoiding DRAM, but this is unconvincing due to the lack of quantitative verification. To show the benefits of BRAM design, comparison with when DRAM is used and discussion of bandwidth constraints are necessary. Third, HDR processing is generally performed on CPUs and ASICs, and a simple comparison with these in terms of performance and flexibility is desirable. In light of the above points, we would like to ask you to add experimental data and discussion.

Strengths:

- The proposed BRAM usage model is well formulated, practically useful, and helps significantly reduce design iteration time in FPGA-based HDR processing systems.

- The evaluation results demonstrate that the estimation error is small and predictable, which enhances the model's reliability for design exploration.

- The structure of the manuscript is clear, and figures are well prepared.

Weakness:

- The idea of using modeling to guide hardware configuration under resource constraints is sound and commendable.

- However, the architectural design (fully pipelined Mertens-like HDR synthesis with Gaussian/Laplacian pyramids) follows well-known image processing practices and offers limited novelty.
\end{screen}

\begin{screen}
Mandatory:

1. The paper claims that DRAM usage increases latency and implementation cost. However, there is no measurement or estimation of actual system latency, either for the proposed BRAM-only design or in comparison with DRAM-based approaches. Please add comments on quantitative latency or throughput (e.g., clock frequency, cycles per frame, fps at given resolution) to support the claims of real-time capability and low latency. The authors can refer their past paper to show their system's performance.
\end{screen}

Thank you for pointing this out. In response to your comment, we have added a discussion on the system's latency and throughput performance in Sec. 5.5.
To clarify and support our claims on low latency and real-time capability of the proposed BRAM-only architecture, 
we have included the quantitative performance data (clock frequency, latency, and fps at target resolution) in the main text and Table 4 of the revised manuscript. 

\vspace{0.3cm}
\begin{screen}
Mandatory:

2. There is no justification for excluding DRAM besides a qualitative assertion of "latency increase." Please discuss whether DRAM could have been a viable alternative under different scheduling/buffering strategies, and clarify whether memory bandwidth limitations played any role in this design choice. The authors can show the performance difference (bandwidth and latency gaps) between DRAM and BRAM by referring a paper such as the following one.

H. Peng et al., "Binary Complex Neural Network Acceleration on FPGA : (Invited Paper)," 2021 IEEE 32nd International Conference on Application-specific Systems, Architectures and Processors (ASAP), NJ, USA, 2021, pp. 85-92, doi: 10.1109/ASAP52443.2021.00021.
\end{screen}
We have added a discussion in Sec. 1 to clarify the rationale behind excluding DRAM in our design.
Specifically, we now include a comparison of the typical memory bandwidth and latency between DRAM and BRAM, 
citing relevant references including the work by H. Peng et al. [9]. 
This comparison helps justify our decision from a hardware performance perspective.
Furthermore, we briefly discuss that although DRAM could be used under different scheduling or buffering strategies, 
such approaches would introduce additional complexity and control overhead, which may compromise the low-latency requirement of our target application. 
These aspects are now addressed in the revised introduction.


\vspace{0.3cm}
\begin{screen}
Mandatory:

3. The paper lacks any comparison with CPU or ASIC-based HDR implementations, which are more common in industry. Even a brief discussion in the related work or conclusion section would help contextualize the advantages and trade-offs of the FPGA approach.
\end{screen}
We have added a new section (Sec. 5.5) to the revised manuscript, 
where we briefly compare our FPGA-based HDR implementation 
with CPU- and GPU-based approaches.
This section discusses typical performance characteristics (such as latency and 
throughput) based on previous studies and reported benchmarks. 
We highlight that FPGA offers advantages in deterministic low-latency processing 
and parallelism, making it suitable for real-time HDR applications, 
particularly when frame-by-frame responsiveness is critical.


\section*{Other Modifications}
Additional corrections were made, including typos in the text.

\begin{enumerate}
  \item Page 7, left : The word ``8-layers'' has been replaced with ``8-layer''
\end{enumerate}

\end{document}